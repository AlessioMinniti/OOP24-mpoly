\subsection{Martina Ravaioli}

Nonostante questa non fosse la mia prima esperienza nei lavori di gruppo, 
un progetto di questa portata non l'avevo mai fatto, è stata un’esperienza per me molto utile 
sia per l’impegno messo, ripagato anche dai risultati, che per le competenze acquisite 
non solo nella dimestichezza con il linguaggio ma in generale con l’analisi di un problema e la progettazione della sua soluzione. 
Io mi sono occupata di tutte le meccaniche “speciali” di gioco ovvero 
tutte quelle meccaniche che fanno uscire il turno dal suo tipico svolgimento. 
Ritengo di aver fatto un buon uso di pattern e strategie nel risolvere le difficoltà incontrate, 
specialmente nella soluzione della creazione delle carte imprevisti o probabilità con i loro comandi.
Il mio più grande apporto nel lavoro svolto in gruppo è stato durante l’analisi iniziale del dominio dove, 
conoscendo in maniera abbastanza approfondita le meccaniche del gioco originale, ho potuto indirizzare 
le scelte del gruppo verso soluzioni più accurate.
Una mancanza che ho riscontrato è stata la difficoltà nel coordinare il mio lavoro 
con quello degli altri membri attraverso gitHub e i vari branch, in questo però mi è venuto in soccorso il fatto che 
la mia parte di lavoro avesse interazioni abbastanza limitate con le parti dei miei compagni.
Ci sono cose che possono essere migliorate e per mancanza di tempo non sono riuscita ad implementare 
tutte le sfaccettature delle meccaniche speciali che volevo però nel complesso 
posso dire di essere soddisfatta del prodotto finale. 
