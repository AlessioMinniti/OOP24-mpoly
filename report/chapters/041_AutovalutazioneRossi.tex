\subsubsection{Davide Rossi}
Personalmente ritengo di aver affrontato il lavoro di gruppo con professionalità. 
Ho sempre cercato di orientare le mie scelte progettuali e implementative favorendo riusabilità e estendibilità, 
sfruttando le conoscenze apprese nel corso per giudicare il codice da me prodotto.
Ho sempre cercato di immedesimarmi nella realizzazione di un’applicazione in un contesto lavorativo, 
e quindi lavorare con l’ottica che un giorno le mie parti avrebbero potuto subire delle modifiche 
o aggiunte che potrebbero essere fatte anche da un’altra persona. 
Questo è stato per me uno stimolo molto utile per realizzare del codice di qualità e impegnarmi 
per fare una buona progettazione perché credo che questo vada ad impattare tanto sulla facilità 
con la quale si possano realizzare le suddette modifiche. 
Penso di essere stato anche molto bravo a rianalizzare il mio codice, 
identificare i punti di debolezza, e produrre nuove soluzioni più congeniali. 
A volte tuttavia tendo a prediligere soluzioni eccessivamente complesse, 
realizzando componenti che fanno ben più di ciò che è richiesto e ho notato che 
questo si è spesso rivelato un problema perché seppure lo facessi per favorire 
l’espandibilità spesso semplicemente il risultato era un sistema complesso e difficile 
da utilizzare. 
Una direzione più congeniale sarebbe quella di ideare soluzioni che rispondano al 
problema che si sta trattando, e che tuttavia lascino la possibilità di essere modificate 
per il proprio scopo. 
All’interno del gruppo ho assunto un ruolo un po’ centrale e di riferimento per gli 
altri membri. Sebbene io mi sia esclusivamente occupato della parte di progetto a me 
assegnata spesso gli altri ragazzi si rivolgevano a me per pareri ed avevo sempre 
un’idea chiara dello stato generale di avanzamento dei lavori. 
Se si dovesse portare avanti il progetto spenderei maggiori energie per rivedere 
la progettazione dell’architettura del model. Credo che parte della debolezza 
dell’architettura attuale sia dovuta al fatto che i componenti principali hanno 
necessità di scambiare dati tra di loro e per fare ciò  sono state prodotte delle 
soluzioni funzionanti e accettabili che tuttavia avrebbero del margine di miglioramento. 
Penso che rivisitare l’ architettura darebbe la possibilità di costruire un model 
ancora più facile da utilizzare, espandere e più resiliente ai problemi.
