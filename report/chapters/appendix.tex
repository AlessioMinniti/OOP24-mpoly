\appendix
\chapter{Guida utente}
\section{MONOPOLY \- GUIDA UTENTE}
\subsection{INTRODUZIONE E OBIETTIVO DEL GIOCO}
Benvenuto nel tuo Monopoly!\newline
L’obiettivo è semplice: diventare il giocatore più ricco accumulando proprietà, incassando affitti e gestendo con attenzione il proprio patrimonio.
Per farlo, i giocatori si alternano in turni scanditi dal lancio dei dadi, muovendosi sul tabellone, acquistando terreni, costruendo case e interagendo con eventi casuali.\newline
La partita può essere affrontata in due modalità:\newline
\begin{itemize}
    \item Classic Mode: attenta gestione finanziaria, rischio bancarotta.
    \item Infinity Mode: gioco libero, senza eliminazioni per bancarotta.  
\end{itemize}
Il gioco supporta più giocatori, ognuno dei quali sceglie un nome e seleziona un colore tra quelli disponibili.
Il regolamento completo è sempre consultabile tramite l’apposito pulsante "?", disponibile sia nel menu iniziale che durante la partita.
\subsection{AVVIO DEL GIOCO}
All’avvio, il gioco presenta un menu iniziale da cui è possibile:\newline
\begin{itemize}
    \item Avviare una nuova partita
    \item Consultare le regole
    \item Modificare il numero di giocatori
    \item Accedere alle impostazioni (Settings) per selezionare la modalità di gioco
\end{itemize}
\textbf{Scelta dei giocatori}\newline
Ogni partecipante inserisce il proprio nome in corrispondenza di uno dei colori disponibili. Ogni colore rappresenta un token unico sul tabellone e identifica visivamente le proprietà acquistate.\newline
\textbf{File di configurazione\texttt{(config.yml)}:}\newline
Il gioco carica all’avvio il file \texttt{config.yml}, che definisce alcuni parametri fondamentali per la partita. Se il file non viene trovato o contiene errori di formattazione, viene caricata automaticamente una configurazione standard valida.\newline
Esempio di contenuto:\newline
\begin{verbatim}
MIN_PLAYERS: 2
MAX_PLAYERS: 6
NUM_DICE: 2
SIDES_PER_DIE: 6
FONT_NAME: SansSerif
FONT_SIZE: 20
INIT_BALANCE: 2000
RULES_FILE: rules/rules.txt
CARDS_FILE: cards/final_cards.json
DECK_FILE: cards/DeckCard.txt
COLORS: RED, BLUE, ORANGE, GREEN, MAGENTA, CYAN, 
YELLOW, BLACK, LIGHT_GRAY, PINK, DARK_GRAY, GRAY
\end{verbatim}
Modalità di gioco:\newline
\begin{itemize}
    \item Classic Mode: il giocatore viene eliminato se al termine del turno ha saldo negativo.
    \item Infinity Mode: non esistono eliminazioni; anche con saldo negativo il gioco prosegue.
\end{itemize}
\subsection{INTERFACCIA GRAFICA E SCHERMATE PRINCIPALI}
L’interfaccia grafica è suddivisa in aree funzionali:\newline
Tabellone:\newline
\begin{itemize}
    \item Visualizza tutte le caselle, token, proprietà e miglioramenti.
    \item Proprietà acquistate mostrate con badge colorato del proprietario.
    \item Case e alberghi visibili direttamente sulle caselle.
\end{itemize}
Pannello giocatore:\newline
\begin{itemize}
    \item Mostra nome, identificativo numerico e saldo attuale del giocatore attivo.
\end{itemize}
Pannelli centrali:\newline
\begin{itemize}
    \item Azioni contestuali: mostra i pulsanti per eseguire azioni (compra proprietà, compra casa, compra hotel, paga affitto).
    \item Dettaglio carta: visualizza la carta ingrandita della casella con info su affitti, proprietario, valore.
\end{itemize}
Comandi principali:\newline
\begin{itemize}
    \item Tira dadi: tira i dadi e avvia il movimento
    \item Gestione Proprietà: consente la gestione e vendita dei beni posseduti
    \item Termina turno: termina il turno se tutte le azioni obbligatorie sono concluse (pagamento affitti, tiro dei dadi, eventuali effetti speciali)
    \item "?": visualizza il regolamento
\end{itemize}
\subsection{SVOLGIMENTO DEL TURNO}
Ogni turno si compone di:
\begin{enumerate}
    \item Fase pre-tiro (facoltativa): vendita di beni tramite "Gestione Proprietà".
    \item Tiro dei dadi: clic su "Tira dadi", movimento sul tabellone.
    \item Interazione con la casella: acquisto, pagamento affitto, carte evento, tasse, acquisti di case/hotel se applicabili.
    \item Consultazione dettagli: tramite pannello carta ingrandita.
    \item Fine turno: clic su "Termina turno"; in Classic Mode è obbligatorio avere saldo positivo per non perdere.
\end{enumerate}
\subsection{PROPRIETÀ E MIGLIORAMENTI}
Acquisto:\newline
\begin{itemize}
    \item Si possono comprare proprietà non possedute da altri giocatori.
\end{itemize}
Affitti:\newline
\begin{itemize}
    \item Si pagano tramite la banca.
    \item Raddoppiati se il proprietario ha l’intero gruppo colore.
    \item Aumentano con case e alberghi.
\end{itemize}
Miglioramenti con case e hotel:\newline
\begin{itemize}
    \item Per poter costruire, è necessario possedere tutte le proprietà dello stesso gruppo colore.
    \item Per poter acquistare case (o hotel), devi essere sopra la proprietà che vuoi migliorare.
    \item Per costruire l’hotel, devi prima avere 4 case sulla proprietà.
\end{itemize}
Vendite:\newline
\begin{itemize}
    \item Case e Hotel: restituiscono il 50\% del valore di acquisto.
    \item Proprietà: vendibili solo se prive di case/hotel, restituiscono il 50\% del valore di acquisto.
    \item Stazioni: restituiscono il 75\% del valore di acquisto.
\end{itemize}
\subsection{PRIGIONE, TASSE, SOCIETA' ED EVENTI SPECIALI}
Prigione:\newline
\begin{itemize}
    \item Inviato se: si finisce sulla casella "Go to Jail" o si pesca la relativa carta imprevisti/probabilità.
    \item Si rimane max 3 turni, si può uscire prime se si ottiene un numero doppio ai dadi durnate il proprio turno.
    \item Durante il proprio turno rimane possibile la gestione delle proprie proprietà tramite l'apposito menù.
\end{itemize}
Tasse:\newline
\begin{itemize}
    \item Tax e carte evento possono richiedere pagamenti alla banca.
\end{itemize}
Imprevisti/Probabilità:\newline
\begin{itemize}
    \item 20 carte con effetti variabili (bonus e malus).
    \item Vengono estratte random e reinserite nel mazzo.
\end{itemize}
Free Parking:\newline
\begin{itemize}
    \item Nessun effetto economico.
    \item Si salta il turno successivo.
\end{itemize}
SOCIETÀ:
\begin{itemize}
    \item Hanno un costo di affitto pari al valore dei dadi moltiplicato per l'affitto base
\end{itemize}
\subsection{FINE DEL GIOCO E CONDIZIONI DI VITTORIA}
Classic Mode:\newline
\begin{itemize}
    \item Eliminazione se saldo negativo a fine turno.
    \item Vince l’ultimo giocatore rimasto.
\end{itemize}
Infinity Mode:\newline
\begin{itemize}
    \item Nessuna eliminazione, anche con saldo negativo.
\end{itemize}
I giocatori possono sempre terminare la partita utilizzando la "X" della finestra di gioco, in quel caso vince il giocatore con il saldo più alto.
\subsection{RIEPILOGO COMANDI E STRUTTURA TECNICA}
Pulsanti principali:
\begin{itemize}
    \item Tira dadi: tira i dadi
    \item Gestione Proprietà: gestisci beni posseduti
    \item Termina turno: termina turno
    \item "?": visualizza regolamento
\end{itemize}
Azioni contestuali (casella attuale):\newline
Buy Property, Buy House, Buy Hotel, Pay Rent, Sell House, Sell Hotel, Sell Property\newline
Regolamento:\newline
\begin{itemize}
    \item Sempre accessibile tramite "?" in qualsiasi momento.
    \item Caricato dal percorso indicato nel campo RULES\_FILE.
\end{itemize}
File config.yml:\newline
Parametri personalizzabili: numero giocatori, dadi, saldo iniziale, font, percorsi ai file, colori dei token dei giocatori
\chapter{Esercitazioni di laboratorio}
\section{davide.rossi47@studio.unibo.it}
\begin{itemize}
    \item Laboratorio 07: \url{https://virtuale.unibo.it/mod/forum/discuss.php?d=177162#p246183}
    \item Laboratorio 08: \url{https://virtuale.unibo.it/mod/forum/discuss.php?d=178723#p247214}
    \item Laboratorio 09: \url{https://virtuale.unibo.it/mod/forum/discuss.php?d=179154#p248357}
    \item Laboratorio 10: \url{https://virtuale.unibo.it/mod/forum/discuss.php?d=180101#p249538}
\end{itemize}
\section{alessio.minniti@studio.unibo.it}
\begin{itemize}
    \item Laboratorio 08: \url{https://virtuale.unibo.it/mod/forum/discuss.php?d=178723#p247295}
    \item Laboratorio 09: \url{https://virtuale.unibo.it/mod/forum/discuss.php?d=179154#p248022}
    \item Laboratorio 10: \url{https://virtuale.unibo.it/mod/forum/discuss.php?d=180101#p249462}
    \item Laboratorio 11: \url{https://virtuale.unibo.it/mod/forum/discuss.php?d=181206#p250861}
\end{itemize}