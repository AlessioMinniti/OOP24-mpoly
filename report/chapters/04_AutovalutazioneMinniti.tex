\subsection{Alessio Minniti}

Nell'ambito del progetto mi sono occupato principalmente di tutta la parte relativa alla gestione del tabellone di gioco, dei turni e della gestione delle case. 
Alla fine sono felice del lavoro svolto anche se col senno di poi avrei cercato di suddividere meglio le parti del gioco 
in quanto durante lo sviluppo si è notato come certe parti risultassero più corpose di altre e quindi talvolta abbiamo riallinearci meglio in seguito.
Tuttavia nonostance ciò mi ritengo soddisfatto, era la prima volta che lavoravo ad un progetto di questo tipo ed ho provato ad assumere un approccio più professionale possibile,
ho dato molta importanza alla struttura della mia parte di model cercando di ottenere un'architettura ben efficientemente struttura e scalabile.
Infatti durante la creazione del gioco ho fatto buon uso di diversi pattern per risolvere alcuni dei problemi in cui mi ero imbattuto.  
A livello di scalabilità del gioco, ho cercato di renderlo adattabile anche per future versioni, come i dadi che sono stati creati in modo che in un futuro 
si possano usare anche più di 2 dadi e con quante facce uno voglia invece delle solite 6, 
oppure ho previsto nella factory anche una creazione delle pawn con una shape diversa da quella di default in caso in cui 
in future versioni si voglia far scegliere la forma della pedina ai giocatori.
Dovessimo portare avanti il progetto cercherei di ripensare meglio alla struttura del gioco per renderla più semplice, perchè credo che un punto debole del gioco
sia il fatto che a volte i componenti principali necessitano di scambiarsi continuamente delle informazioni e nonostante abbiamo trovato soluzioni ottimali
ritengo che si potrebbe migliorare.